\documentclass[11pt,a4paper]{article} % ein Artikel in 11-Punkt Schrift
% wie man sich schon denkt leitet % einen Kommentar bis Zeilenende ein

\usepackage[german]{babel} % deutsch, deutsche Rechtschreibung
\usepackage[utf8]{inputenc} % Unicode Text 
\usepackage[T1]{fontenc} % Umlaute und deutsches Trennen
\usepackage{mathptmx} % Times New Roman, gewohnter Font
\usepackage{courier} % Schreibmaschinenfont schicker
\usepackage[scaled=.95]{helvet} % was serifenloses wenn gebraucht
\usepackage{graphicx} % wir wollen Bilder einfügen

\usepackage{listings} % Schöne Quellcode-Listings
\lstset{basicstyle=\sffamily, columns=[l]flexible, mathescape=true, 
  showstringspaces=false, numbers=left, numberstyle=\tiny}
\lstset{language=python} 
% und eine eigene Umgebung für Listings
\usepackage{float}
\newfloat{listing}{htbp}{scl}[section]
\floatname{listing}{Listing}

% Auch wenn es anrüchig ist, man kann den Platz etwas mehr ausnützen
\usepackage[paper=a4paper,width=14cm,left=35mm,height=22cm]{geometry}
\usepackage{setspace}
\linespread{1.10} % nicht ganz anderthalbzeilig, nur ein bisschen mehr Platz
\setlength{\parskip}{0.5em} % kleiner Paragraphenabstand
\setlength{\parindent}{0em} % im Deutschen Einrückung nicht üblich, leider

% Seitenmarkierungen 
\usepackage{fancyhdr} % Schickere Header und Footer
\pagestyle{fancy}
% font für Header/Footer
\newcommand{\phv}{\fontfamily{phv}\fontseries{m}\fontsize{9}{11}\selectfont}
\fancyhead[L]{\phv Praktikumsbericht} 
\fancyhead[R]{\phv \thepage}
\fancyfoot[L]{\phv Aoe GmbH}
\fancyfoot[C]{\ } % keine Seitenzahl unten
\fancyfoot[R]{\phv Michael Sandritter}

% Ein spezielles Paket zum Aufteilen des Literaturverzeichnisses
\usepackage{bibtopic}

\title{Praktikumsbericht}
\author{Michael Sandritter \\ AOE GmbH}

\date{\today}

\begin{document}
\maketitle % erzeugt den Titel mit Autor und Datum


\newpage % neue Seite, muss bei einem Artikel eigentlich nicht sein

\section{Praktikumsbetrieb} \label{sec:betrieb} 
% mit \label können wir die Einführung referenzieren

Mein Praktikum habe ich bei der AOE GmbH in Wiesbaden absolviert. 
Die AOE GmbH hat seinen Hauptsitz in Wiesbaden und ist Dienstleister für Open Source Enterprise Lösungen. 
AOE wurde 1999 unter dem Firmennamen AOE media gegründet und arbeitete anfänglich als TYPO3 Dienstleister. 
Inzwischen umfassen die Serviceleistungen Open Source Web Portale, E-Commerce und mobile Anwendungen, 
die für globale agierende Unternehmen entwickelt werden. 
Das Kollegium umfasst 180 Entwickler und Consultants an 8 Standorten, davon 120 Mitarbeiter in der Zentrale in Wiesbaden. 
Die Unternehmenskultur setzt auf die Kernelemente: Open Source, objektorientierte Programmierung und Methoden 
wie Agile Software-Entwicklung und Test-Driven-Development. 

Dabei pflegt die AOE GmbH eine dezentrale Unternehmensstruktur. Jedem Kunden wird ein Team aus Entwicklern zur Seite gestellt. 
Das Team arbeitet von dort ab komplett selbstständig und setzt alle Projektphasen in engem Kontakt mit dem Kunden um.

\section{Arbeitsumfeld} \label{sec:umfeld}

In meinem Praktikum wurde ich vorwiegend als Backend-Entwickler im Congstar-Team eingesetzt. 
Das Team bestand dabei aus 19 Entwicklern und acht Testern, die sich ihrerseits wieder in 3 kleinere Teams aufteilen. 
Die Frontend bzw. Backend-Kompetenzen waren dabei gleichmäßig auf die verschiedenen Teams verteilt,
 so dass jedes Team, für sich, in der Lage war sowohl Backend- als auch FrontendßTasks zu übernehmen.

Jedes Team arbeitet nach der agilen Softwareentwicklungsmethodik “Scrum”. 
In täglichen Daily Meetings tauschen sich die Entwickler in ihren kleinen Teams über den aktuellen Entwicklungsstand aus, 
indem jeder Entwickler und Tester, dem Team erzählt, woran er gerade arbeitet und gegebenenfalls 
schildert welche Probleme bei der Umsetzung existieren.
Zusätzlich finden zwei mal wöchentlich Weekly Meetings statt, 
bei denen alle Entwickler und Tester zu einem übergreifenden Update des Entwicklungsstand zusammenkommen. 
In der großen Runde werden Themen besprochen, die das komplette Team betreffen. 
Dazu gehört zum Beispiel, das Einführen und Verwenden neuer Technologien, das Entwickeln oder 
Ändern der Software Architektur und generelle organisatorische Themen.

Der zeitliche Rahmen gibt vor, dass dem Kunden alle zwei Monate ein neues Softwarepaket mit neuen Features ausgeliefert wird. 
Innerhalb dieser Zeit absolviert, jedes Team drei Sprints. Die ersten beiden Sprints sind Feature Sprints, 
die jeweils 3 Wochen Arbeitszeit umfassen. In diesen beiden Sprints werden neue Stories umgesetzt, 
die davor von den Product Ownern (PO) priorisiert worden sind. Das Team entscheidet jedoch für jeden Sprint neu, 
wie viele Stories realistisch umsetzbar sind und arbeitet zu jeder Story kleinere Tasks aus, 
die von den Entwicklern “geclaimed” und umgesetzt werden. Der letzte Sprint innerhalb der zwei Monate ist ein Release Sprint. 
Dieser erstreckt sich über die beiden verbleibenden Wochen. 
Zu dieser Zeit finden Verbundtests statt, bei denen die neu entwickelten Features im Zusammenspiel mit der $Aax^{2}$ getestet wird. 

Die $Aax^{2}$ ist die Software aus dem Hause Compax. Diese hält den congstar-spezifischen Produktkatalog. 
Darin werden alle Congstar Produkte abgebildet, wie zum Beispiel die Post- und Prepaidtarife, mit den dazugehörigen Buchungsoptionen. 
Die $Aax^{2}$ bietet dafür Soap-Schnittstellen an, um die Produktdaten im, von AOE entwickelten, “Backend” beziehen zu können. 

Infolge des Release Sprints, besteht für die Entwickler ein Entwicklungs-Stop, 
so dass das neue Softwarepaket im Verbundtest getestet werden kann. Anstatt an neuen Features weiter zu arbeiten, 
wird die Zeit einerseits zum refactorn der Software und zum anderen zum update von Frameworks und anderer Software genutzt.

Am Ende jedes Sprints findet zusammen mit den Stakeholdern das Review statt, 
bei dem das Entwickler-Team die Ergebnisse des aktuell endenden Sprints präsentieren. 
Dafür fahren die Entwickler-Teams nach jedem Feature-Sprint nach Köln zu Congstar, 
während sich die Beteiligten nach Ende des Release-Sprints bei Aoe in Wiesbaden treffen. 
Hauptsächlich geht es um die Präsentation der Stories, die innerhalb des Sprints erfolgreich umgesetzt werden konnten. 
Das Entwickler-Team berichtet in diesem Zusammenhang was während des Sprints gut gelaufen ist, welche technischen Probleme und 
Hindernisse sich ergeben haben und wie diese Probleme gelöst wurden. Die Stakeholder haben in diesem Zuge 
die Möglichkeit offene Fragen mit den Entwicklern direkt zu klären. Alle Beteiligten erhalten somit 
einen Überblick über den aktuellen Stand des Projekts, der sich im Produkt-Backlog widerspiegelt. 
Das Review dient somit als Grundlage für das darauf folgende Sprint Planning.

Im Sprint Planning wird im ersten Teil besprochen, welche Stories und wie viele Stories 
im Laufe des nächsten Sprints umgesetzt werden können. Während im zweiten Teil besprochen wird, 
wie diese Stories umgesetzt werden sollen.
Dies geschieht im Zusammenspiel von Entwickler-Team, PO und Scrum Master. Gemeinsam priorisieren sie das Sprintziel. 
Dafür werden die Stories im Backlog priorisiert und für jede Story eine Aufwandsschätzung durch die Entwickler abgegeben. 
Parallel wird grob berechnet wie groß die “Velocity” des Teams ist, um einordnen zu können wie viele Stories 
innerhalb eines Sprints tatsächlich umgesetzt werden können. Nachdem die Stories für den nächsten Sprint festgelegt sind, 
definieren die Entwickler für jede Story kleinere Tasks. 

Backlog, Stories und Tasks werden im Projekt-Management-Tool
zugänglich gemacht. Aktuell wird als solches Tool Kunagi verwendet. Künftig soll Kunagi jedoch durch JIRA ersetzt werden.  

Im Laufe des Sprints werden die einzelnen Tasks von den Entwicklern abgearbeit. Diese müssen dabei vermerken, 
wie viel Zeit sie für den Task “geburned” sprich benötigt haben. Aus der geschätzen und der letztendlich benötigten Zeit 
lässt sich ein Burndown-Chart ermitteln, das den aktuellen Fortschritt des Sprints wiederspiegelt.

Sind alle Tasks einer Story umgesetzt steht die Abnahme dieser Story an. Diese erfolgt 
durch einen Entwickler der an der Umsetzung der Story beteiligt war und durch den Product Owner. 
Der Entwickler präsentiert dem PO die Features, die durch die Story umgesetzt werden sollten. 
Erfüllt das Umgesetzte alle Akzeptanz-Kriterien, die für die Story definiert wurden, 
kann die Story durch den PO abgenommen werden und ist fertig.

\section{Projektbeschreibung} \label{sec:projekt}

Entwickelt wird der Webshop der Firma Congstar. Congstar ist als Schwestergesellschaft der Telekom, 
deutschlandweiter Anbieter für Mobilfunk- und DSL-Produkte. 

Dem Kunden soll zum einen abhängig von seinem Endgerät alle relevanten Informationen zum Congstar Portfolio und 
Service dargestellt bekommen und unabhängig von seinem Endgerät, 
neue Verträge abschließen und bestehende Verträge verwalten können. 

Für die Redakteure von Congstar sollen spezielle Informationen des Webshops redaktionell und zentral pflegbar sein, 
um inhaltliche Inkonsistenzen und redundante Informationen zu vermeiden. 

\section{Deployment Pipeline} \label{sec:pipeline}

Im folgenden wird grob der Ablauf der Deployment-Pipeline erl{\"a}utert. 
\begin{figure}[h]
\includegraphics[width=\textwidth]{images/DeploymentPipeline.pdf}
\centering
\end{figure}



\newpage

% Listen, wenn überhaupt!, bitte ans Ende und nicht an den Anfang
%\listoffigures % Liste der Abbildungen 
%\listoftables % Liste der Tabellen

\newpage

% Als letztes noch das Literaturverzeichnis
\bibliographystyle{plain}
% so wäre es ganz einfach!
%\bibliography{ausarb,online}
% dann mit "bibtex ausarb" bibtexen und das Literaturverzeichnis ist da

% z.B. mit bibtopic kann man die Quellen sauber trennen
\begin{btSect}{ausarb}
\section*{Literaturverzeichnis}
\btPrintCited
\end{btSect}
\begin{btSect}{online}
\section*{Online-Quellen}
\btPrintCited
\end{btSect}
% dann mit "bibtex ausarb1" und "bibtex ausarb2" arbeiten
% Wir verwenden ausarb<i> weil die Dokumenten-Datei ausarb.tex ist

\end{document}
